%% ## Construye tu propia portada ##
%% 
%% Una portada se conforma por una secuencia de "Blocks" que incluyen
%% piezas individuales de informaci'on. Un "Block" puede incluir, por
%% ejemplo, el t'itulo del documento, una im'agen (logotipo de la universidad),
%% el nombre del autor, nombre del supervisor, u cualquier otra pieza de
%% informaci'on.
%%
%% Cada "Block" aparece centrado horizontalmente en la p'agina y,
%% verticalmente, todos los "Blocks" se distruyen de manera uniforme 
%% a lo largo de p'agina.
%%
%% Nota tambi'en que, dentro de un mismo "Block" se pueden cortar
%% lineas usando el comando \\
%%
%% El tama'no del texto dentro de un "Block" se puede modificar usando uno de
%% los comandos:
%%   \small      \LARGE
%%   \large      \huge
%%   \Large      \Huge
%%
%% Y el tipo de letra se puede modificar usando:
%%   \bfseries - negritas
%%   \itshape  - it'alicas
%%   \scshape  - small caps
%%   \slshape  - slanted
%%   \sffamily - sans serif
%%
%% Para producir plantillas generales, la informaci'on que ha sido inclu'ida
%% en el archivo principal "tesis.tex" se puede accesar aqu'i usando:
%%   \insertauthor
%%   \inserttitle
%%   \insertsupervisor
%%	 \insertcorreferent
%%   \insertinstitution
%%   \insertdegree
%%   \insertfaculty
%%   \insertdepartment
%%	 \insertcity
%%	 \insertcountry
%%   \insertsubmitdate

\begin{titlepage}
	\TitleBlock[\vspace{-1.5 in}]{
		\includegraphics[height=4cm]{Imagenes/Portada/logo_universidad}}
	\TitleBlock[\vspace{0.00in}]{
		\bfseries\insertinstitution}
	\TitleBlock[\vspace{0.00in}]{
		\insertdepartment}
	\TitleBlock[\vspace{0.00in}]{
		\insertcity - \insertcountry}
	\TitleBlock[\vspace{1.0in}]{
		\small\scshape\bfseries\inserttitle}
    \TitleBlock[\vspace{1.0in}]{
    	Tesis de grado presentada por\\
    	\textbf{\insertauthor} \\
    	como prerequisito parcial para optar al grado de \\
    	\textbf{\insertdegree}
		\begin{minipage}{0.45\textwidth}
			\TitleBlock[\vspace{1in}]{
				\begin{center}
					Profesor Guia \\
					\insertsupervisor
				\end{center}}
		\end{minipage}
		\begin{minipage}{0.45\textwidth}
			\TitleBlock[\vspace{1in}]{
				\begin{center}
					Profesor Correferente \\
					\insertcorreferent
				\end{center}}
		\end{minipage}}
	%\TitleBlock[\vspace{1.5in}]{
	%\insertsubmitdate}
\end{titlepage}

%% Nota 1:
%% Se puede agregar un escudo o logotipo en un "Block" como:
%%   \TitleBlock{\includegraphics[height=4cm]{escudo_uni}}
%% y teniendo un archivo "escudo_uni.pdf", "escudo_uni.png" o "escudo_uni.jpg"
%% en alg'un lugar donde LaTeX lo pueda encontrar.

%% Nota 2:
%% Normalmente, el espacio entre "Blocks" se extiende de modo que el
%% contenido se reparte uniformemente sobre toda la p'agina. Este
%% comportamiento se puede modificar para mantener fijo, por ejemplo, el
%% espacio entre un par de "Blocks". Escribiendo:
%%   \TitleBlock{Bloque 1}
%%   \TitleBlock[\bigskip]{Bloque2}
%% se deja un espacio "grande" y de tama~no fijo entre el bloque 1 y 2.
%% Adem'as de \bigskip est'an tambi'en \smallskip y \medskip. Si necesitas
%% aun m'as control puedes usar tambi'en, por ejemplo, \vspace*{2cm}.


