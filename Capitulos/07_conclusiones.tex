\chapter{Conclusiones y Trabajo Futuro}
De todo el trabajo realizado, y considerando los objetivos planteados al comienzo de este documento, se extraen las siguientes conclusiones:

\begin{itemize*}
	\item El proceso de asimilación de datos 4D multipunto en la capa límite planetaria aplicado a un modelo atmósferico simulando a alta resolución y utilizando LES no demostró una ventaja comparativa con respecto a las técnicas de simulación ya existentes. La justificación de esto está en el empeoramiento de las métricas estadísticas seleccionadas. Las razones para esto incluyen diversos factores, desde la selección de las condiciones de borde, la manera de anidar dominios, los modelos de las físicas, etc. Sin embargo, y considerando la calidad de los resultados obtenidos para el terreno plano, gran parte de la discrepancia posiblemente sea consecuencia de la interacción del flujo con el terreno complejo. Bajo ese criterio entonces, podemos asumir que las variables clave para mejorar los resultados realizados serán la resolución de la malla y el modelo de turbulencia usado.
	\item Con respecto a la asimilación de datos en si, se demostró que a pesar de tener pocos puntos y a muy baja altura, en comparación a la altura del borde superior del dominio, esta puede provocar diferencias significativas con respecto a los campos no asimilados. Esta diferencia es notoria para el caso multipunto, siendo en el caso puntual una diferencia no tan apreciable. Para terreno plano se demostró que una asimilación puntual significó mejoras al contrastar los resultados con datos reales, sin embargo en terreno complejo una asimilación multipunto ensució fuertemente los resultados.
	\item Con respecto a la utilización del LES y al acople de la meso con la microescala, esta fue realizada exitosamente. La forma de los espectros en los niveles inferiores y las variables de segundo orden exponen que el LES se desarrollo de buena manera. Llama la atención el buen comportamiento de las variables, considerando que las simulaciones realizadas son reales, no teniendo los beneficios que presentan las simulaciones ideales presentes en la literatura como condiciones de borde periódicas o campos impuestos de energía cinética turbulenta confirmando el por qué el modelo WRF sigue siendo un modelo ampliamente usado. Para el caso complejo, donde se tiene una malla de resolución de casi 3 [m], las componentes resueltas del campo turbulento se pueden apreciar en la serie de tiempo, sin embargo el modelo aún es incapaz de representar la ráfagas, probablemente debido a la excesiva difusividad de la clausura turbulenta.
	\item Con respecto al desarrollo de códigos, varias mejoras y utilidades fueron implementadas exitosamente al modelo manualmente para optimizar el output de los resultados y las series de tiempo. La programación del modo cíclico con el cual se ejecuto el modelo con DA también fue en parte exitoso, sin embargo presenta la falencia que reinicia los campos de energía cinética turbulenta perjudicando el periodo de \emph{spinup} del modelo.
	\item Las bases de datos reales de alta resolución seleccionadas fueron implementadas satisfactoriamente al modelo. Las no linealidades manifestadas en el caso complejo no hubiesen aparecido si fuese de otra forma. De esta manera se destaca la importancia absoluta de tener bases de datos orográficas y de uso de suelo actualizadas y a muy alta resolución. Es necesario impulsar campañas de medición para tener datos de manera libre y así continuar la investigación sobre terreno complejo. La información de uso de suelo también afecta grandemente los resultados de la simulación, encontrándose diferencias de hasta 3 [m/s] en la rapidez media del viento a cambios del valor de $z_0$\footnote{Pequeños análisis de sensibilidad fueron realizados a lo largo de la investigación que demuestran este hecho, pero que se omiten en este documento. Sin embargo este hecho está documentado numerosamente en la literatura.}.
	\item En temas de validación, los resultados en terreno plano demostraron ser precisos para manifestar los comportamientos presentes en la literatura, y por lo tanto se valida la metodología propuesta. Para los valores en terreno complejo es difuso dar una conclusión certera ya que, por un lado, los datos medidos son pocos y, por otro, los resultados para la comparación ciega entregan un espectro amplio de valores. Sin embargo, a grandes rasgos, se rescata la tendencia que debería existir en el dominio, en términos de aceleración y separación, y los valores obtenidos no están lejos de aquellos medidos en terreno.
\end{itemize*}
\section{Trabajo Futuro}
El alcance que posee este trabajo es bastante amplio, por lo tanto es posible implementar mejoras y realizas investigación en muchos aspectos. A continuación se listan algunos, los mas relevantes para el autor, para continuar el trabajo y seguir buscando mejoras a este acercamiento:
\begin{itemize*}
	\item La condición de borde superior del modelo está fijada a una presión constante de $30000$ [Pa] que corresponderían a una altura aproximada de 9 [km]. Este valor está limitado debido a las condiciones iniciales y de borde que provienen de un modelo global. El área de interés para la industria eólica (y la humanidad en general) está en los primeros 2 [km] de atmósfera aproximadamente, lo que conlleva a que muchos elementos de la malla numérica están siendo resueltos y no aportan a las soluciones que se están buscando. Una manera de sobrellevar esto sería la aplicación de un anidamiento vertical a medida que se anidan los distintos dominios. Esta técnica aún está siendo probada por la comunidad científica y se encuentra en implementación experimental, pero es relevante para los objetivos planteados analizar si su uso presenta beneficios para la metodología propuesta.
	\item Debido a las limitantes de las campañas de medición con las que se compararon los resultados, fue imposible analizar la sensibilidad de la asimilación puntual y multipunto para el mismo caso. Es necesario hallar o crear nuevas bases de datos que permitan asimilación de datos multipunto en terreno complejo y analizar los resultados. Bajo esta mismo observación, también es necesario probar la utilización de otros métodos de asimilación de datos (de ensamble, filtro de Kalman, etc.) o sensibilizar con los parámetros característicos del DA que se encarga de ponderar la matriz $\textbf{B}$.
	\item Siguiendo con las mejoras al esquema de asimilación, es necesario probar esta con la utilización de una cantidad masiva de puntos en el dominio (tal cual como se hace para escalas sinópticas que presentan buenos resultados). Para esto es necesario la existencia de campañas de medición locales con instrumentación de vanguardia. Se espera que a futuro, los globos meteorológicos descritos en la introducción de esta tesis, estén finalizados y se puedan llevar a cabo experimentos mas concretos y locales.
	\item Con respecto a la programación y el código del modelo, se debe implementar un algoritmo cíclico que permita la continuidad del campo de energía cinética turbulenta. Una solución a esto podría ser ejecutar el código WRF luego de la asimilación en su modo de reinicio, sin embargo se desconoce el comportamiento del paquete de asimilación de datos con el archivo de outputs de \emph{restart}.
	\item Considerando las conclusiones emanadas con respecto a que los aspectos críticos para la simulación son aquellos relacionadas a la interacción del terreno complejo y la turbulencia, es necesario probar el comportamiento del modelo con otras parametrizaciones para las físicas, en específico, para el modelo de suelo y la capa superficial. Se podría considerar el uso de esquemas de mayor orden o aquellos de nueva generación, que quizás, por un lado aumenten el esfuerzo computacional de la simulación, pero que entreguen mejores resultados.
	\item En la misma línea de lo anterior, se deben probar otros esquemas para la parametrización de la turbulencia, tanto para los dominios de meso, como para aquellos de microescala. En la microescala se podrían implementar esquemas para la clausura turbulenta de orden superior, como un modelo LES dinámico u otro que incluya el \emph{backscatter}.
	\item Todas las simulaciones realizadas significaron un costo computacional demasiado elevado como para ser utilizado en ambientes operacionales (ver Apéndice \ref{apB}), una manera de disminuir esto sería con la incorporación de un paso de tiempo variable en función de las condiciones de inestabilidad del modelo, sin embargo esto podría ocasionar problemas a la hora de asimilar datos en horas específicas.
	\item Es sabido que para un correcto LES, las zonas de interés deben estar lejos de los bordes del dominio para evitar la influencia de estos en la generación de TKE. Para esta investigación debido a que se usaron las bases de datos entregadas por los desarrolladores, fue imposible ampliar la malla de forma que la colina de Bolund quedara lejos de los bordes. Se podría evaluar hacer una manipulación a las bases de datos para evitar esto y así analizar la influencia que tuvo este aspecto en los resultados presentados acá.
	\item Finalmente, sería conveniente probar el modelo utilizando sistemas de coordenadas vanguardistas, como por ejemplo el método de la frontera inmersa o una coordenada híbrida para la vertical. Las últimas investigaciones con respecto a estos esquemas han tenido un relativo éxito y por lo tanto basta su implementación para investigarlos en el contexto de la alta resolución y el LES. A la fecha, el modelo WRF va en su versión 4.1 y en su versión 3.9 ya implementó la coordenada híbrida para la presión.
\end{itemize*}
\newpage
\section{Palabras Finales}
Bajo el espíritu solidario que está en el núcleo del desarrollo de esta tesis, todos los códigos de las figuras, archivos de configuración, códigos de utilidad y algunos resultados se encuentran de manera pública en el Github del autor\footnote{\url{https://github.com/aababbbba/4.plots}}.