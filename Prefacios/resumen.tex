\prefacesection{Resumen}
Con el fin de lograr una correcta predicción del recurso viento en terreno complejo para zonas muy localizadas, se llevaron cabo una serie de simulaciones numéricas meteorológicas multiescala con datos reales de alta resolución utilizando el modelo atmosférico WRF y una clausura LES para la turbulencia.
El acoplamiento de las meso y microescala se logró a través de la técnica de dominios anidados hasta llegar a una resolución de aproximadamente 2 [m]. Para corregir las desviaciones e incertidumbres propias de una simulación numérica, se propuso utilizar un esquema de asimilación de datos cuatridimensional en el dominio mas interior.

La validación de la implementación se basa en 4 casos. Los primeros dos corresponden a una simulación en el sitio de pruebas de turbinas en Høvsøre, Dinamarca, el cual es un terreno cuasi-plano ampliamente estudiado. La primera simulación valida el acercamiento numérico y la segunda muestra la influencia de la asimilación de datos en la capa límite considerando 5 niveles de un mástil meteorológico ubicado en el centro del dominio.

Los últimos dos casos corresponden a la aplicación de la misma metodología en terreno complejo para simular la colina de Bolund ubicada también en Dinamarca. Estas dos simulaciones exponen: (i) el comportamiento del modelo para un flujo de viento neutralmente estratificado con burbuja de separación y, (ii) la influencia de la asimilación de datos multipunto utilizando la información de 8 mástiles en 3 niveles cercanos a la superficie.

Los resultados obtenidos muestran que es posible obtener predicciones más certeras y que replican el comportamiento turbulento del viento a las escalas simuladas y que, además, la asimilación de datos mejora esta predicción en el caso de terreno plano en un 10\%. En terreno complejo, la asimilación de datos no logra mejorar la solución debido a la cercanía de las mediciones con el suelo y los forzamientos inducidos por el terreno, sin embargo toda la investigación da pié a un uso operativo de la metodología y los códigos propuestos.

\paragraph{Palabras Clave} \emph{Simulación Multiescala, LES, Asimilación de Datos, WRF, Capa Limite Atmosférica, NWP, Turbulencia Atmosférica, Energía Eólica}