\prefacesection{Resumen}
Con el fin de estudiar el rendimiento del modelo numérico ARW-WRF para captar la dinámica del viento en terreno complejo, se realizaron simulaciones meso-microescala en Punta Curaumilla, lugar de reconocido potencial eólico chileno a una resolución de hasta 50 [m] en la malla horizontal. Las simulaciones se llevaron a cabo utilizando la técnica de dominios anidados de una vía, bases de datos de alta resolución e incorporando Large Eddy Simulation en los dominios mas pequeños con el uso de un modelo de clausura 1.5 TKE. Los resultados obtenidos presentan grandes avances en la caracterización del viento en la zona y demuestra que es posible exigir una mayor resolución para los mapas de viento actuales.