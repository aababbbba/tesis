%% Las secciones del "prefacio" inician con el comando \prefacesection{T'itulo}
%% Este tipo de secciones *no* van numeradas, pero s'i aparecen en el 'indice.
%%
%% Si quieres agregar una secci'on que no vaya n'umerada y que *tampoco*
%% aparesca en el 'indice, usa entonces el comando \chapter*{T'itulo}
%%
%% Recuerda que aqu'i ya puedes escribir acentos como: 'a, 'e, 'i, etc.
%% La letra n con tilde es: 'n.
\prefacesection{Abstract}
In order to achieve a correct wind resource prediction in complex terrain for localized areas, a series of multiscale meteorological numerical simulations with high resolution real data was carried out using the WRF atmospheric model and a LES turbulence closure.
The meso-microscale coupling was done through nested domains until a resolution of approximately 2 [m] was reached. To correct the numerical simulation deviations and uncertainties, a four-dimensional data assimilation scheme in the innermost domain was proposed. 

The implementation's validation is based in 4 cases. The first two correspond to a simulation at the turbine test site in Høvsøre, Denmark, which is a widely studied quasi-flat terrain. The first simulation validates the numerical approach and the second one shows the influence of the data assimilation in the boundary layer considering 5 levels of a meteorological mast located at the domain center.

The last two cases correspond to the application of the same methodology but in complex terrain to simulate the Bolund hill, also located in Denmark. These two simulations exposes: (i) the behavior of the model for a neutrally stratified wind flow with a separation bubble and, (ii) the multipoint data assimilation influence using information of 8 mast at 3 levels near the surface.

The obtained results shows that it is possible to obtain more accurate predictions that replicate the turbulent wind behavior at simulated scales and that, in addition, data assimilation improves the flat terrain prediction by 10\%.  In complex terrain, the data assimilation fails to improve the solution due to the proximity of the measurements with the ground and the terrain induced forcing. However, all the research gives rise to an operational use of the proposed methodology and codes.

\color{black}
\paragraph{Keywords} \emph{Multiscale Simulation, LES, Data Assimlation, WRF, Atmospheric Boundary Layer, NWP, Atmospheric Turbulence, Wind Energy}
